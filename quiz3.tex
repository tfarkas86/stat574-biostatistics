% Options for packages loaded elsewhere
\PassOptionsToPackage{unicode}{hyperref}
\PassOptionsToPackage{hyphens}{url}
%
\documentclass[
]{article}
\usepackage{lmodern}
\usepackage{amsmath}
\usepackage{ifxetex,ifluatex}
\ifnum 0\ifxetex 1\fi\ifluatex 1\fi=0 % if pdftex
  \usepackage[T1]{fontenc}
  \usepackage[utf8]{inputenc}
  \usepackage{textcomp} % provide euro and other symbols
  \usepackage{amssymb}
\else % if luatex or xetex
  \usepackage{unicode-math}
  \defaultfontfeatures{Scale=MatchLowercase}
  \defaultfontfeatures[\rmfamily]{Ligatures=TeX,Scale=1}
\fi
% Use upquote if available, for straight quotes in verbatim environments
\IfFileExists{upquote.sty}{\usepackage{upquote}}{}
\IfFileExists{microtype.sty}{% use microtype if available
  \usepackage[]{microtype}
  \UseMicrotypeSet[protrusion]{basicmath} % disable protrusion for tt fonts
}{}
\makeatletter
\@ifundefined{KOMAClassName}{% if non-KOMA class
  \IfFileExists{parskip.sty}{%
    \usepackage{parskip}
  }{% else
    \setlength{\parindent}{0pt}
    \setlength{\parskip}{6pt plus 2pt minus 1pt}}
}{% if KOMA class
  \KOMAoptions{parskip=half}}
\makeatother
\usepackage{xcolor}
\IfFileExists{xurl.sty}{\usepackage{xurl}}{} % add URL line breaks if available
\IfFileExists{bookmark.sty}{\usepackage{bookmark}}{\usepackage{hyperref}}
\hypersetup{
  pdftitle={Stat 590 Quiz 3},
  pdfauthor={Tim Farkas},
  hidelinks,
  pdfcreator={LaTeX via pandoc}}
\urlstyle{same} % disable monospaced font for URLs
\usepackage[margin=1in]{geometry}
\usepackage{color}
\usepackage{fancyvrb}
\newcommand{\VerbBar}{|}
\newcommand{\VERB}{\Verb[commandchars=\\\{\}]}
\DefineVerbatimEnvironment{Highlighting}{Verbatim}{commandchars=\\\{\}}
% Add ',fontsize=\small' for more characters per line
\usepackage{framed}
\definecolor{shadecolor}{RGB}{248,248,248}
\newenvironment{Shaded}{\begin{snugshade}}{\end{snugshade}}
\newcommand{\AlertTok}[1]{\textcolor[rgb]{0.94,0.16,0.16}{#1}}
\newcommand{\AnnotationTok}[1]{\textcolor[rgb]{0.56,0.35,0.01}{\textbf{\textit{#1}}}}
\newcommand{\AttributeTok}[1]{\textcolor[rgb]{0.77,0.63,0.00}{#1}}
\newcommand{\BaseNTok}[1]{\textcolor[rgb]{0.00,0.00,0.81}{#1}}
\newcommand{\BuiltInTok}[1]{#1}
\newcommand{\CharTok}[1]{\textcolor[rgb]{0.31,0.60,0.02}{#1}}
\newcommand{\CommentTok}[1]{\textcolor[rgb]{0.56,0.35,0.01}{\textit{#1}}}
\newcommand{\CommentVarTok}[1]{\textcolor[rgb]{0.56,0.35,0.01}{\textbf{\textit{#1}}}}
\newcommand{\ConstantTok}[1]{\textcolor[rgb]{0.00,0.00,0.00}{#1}}
\newcommand{\ControlFlowTok}[1]{\textcolor[rgb]{0.13,0.29,0.53}{\textbf{#1}}}
\newcommand{\DataTypeTok}[1]{\textcolor[rgb]{0.13,0.29,0.53}{#1}}
\newcommand{\DecValTok}[1]{\textcolor[rgb]{0.00,0.00,0.81}{#1}}
\newcommand{\DocumentationTok}[1]{\textcolor[rgb]{0.56,0.35,0.01}{\textbf{\textit{#1}}}}
\newcommand{\ErrorTok}[1]{\textcolor[rgb]{0.64,0.00,0.00}{\textbf{#1}}}
\newcommand{\ExtensionTok}[1]{#1}
\newcommand{\FloatTok}[1]{\textcolor[rgb]{0.00,0.00,0.81}{#1}}
\newcommand{\FunctionTok}[1]{\textcolor[rgb]{0.00,0.00,0.00}{#1}}
\newcommand{\ImportTok}[1]{#1}
\newcommand{\InformationTok}[1]{\textcolor[rgb]{0.56,0.35,0.01}{\textbf{\textit{#1}}}}
\newcommand{\KeywordTok}[1]{\textcolor[rgb]{0.13,0.29,0.53}{\textbf{#1}}}
\newcommand{\NormalTok}[1]{#1}
\newcommand{\OperatorTok}[1]{\textcolor[rgb]{0.81,0.36,0.00}{\textbf{#1}}}
\newcommand{\OtherTok}[1]{\textcolor[rgb]{0.56,0.35,0.01}{#1}}
\newcommand{\PreprocessorTok}[1]{\textcolor[rgb]{0.56,0.35,0.01}{\textit{#1}}}
\newcommand{\RegionMarkerTok}[1]{#1}
\newcommand{\SpecialCharTok}[1]{\textcolor[rgb]{0.00,0.00,0.00}{#1}}
\newcommand{\SpecialStringTok}[1]{\textcolor[rgb]{0.31,0.60,0.02}{#1}}
\newcommand{\StringTok}[1]{\textcolor[rgb]{0.31,0.60,0.02}{#1}}
\newcommand{\VariableTok}[1]{\textcolor[rgb]{0.00,0.00,0.00}{#1}}
\newcommand{\VerbatimStringTok}[1]{\textcolor[rgb]{0.31,0.60,0.02}{#1}}
\newcommand{\WarningTok}[1]{\textcolor[rgb]{0.56,0.35,0.01}{\textbf{\textit{#1}}}}
\usepackage{graphicx}
\makeatletter
\def\maxwidth{\ifdim\Gin@nat@width>\linewidth\linewidth\else\Gin@nat@width\fi}
\def\maxheight{\ifdim\Gin@nat@height>\textheight\textheight\else\Gin@nat@height\fi}
\makeatother
% Scale images if necessary, so that they will not overflow the page
% margins by default, and it is still possible to overwrite the defaults
% using explicit options in \includegraphics[width, height, ...]{}
\setkeys{Gin}{width=\maxwidth,height=\maxheight,keepaspectratio}
% Set default figure placement to htbp
\makeatletter
\def\fps@figure{htbp}
\makeatother
\setlength{\emergencystretch}{3em} % prevent overfull lines
\providecommand{\tightlist}{%
  \setlength{\itemsep}{0pt}\setlength{\parskip}{0pt}}
\setcounter{secnumdepth}{-\maxdimen} % remove section numbering
\ifluatex
  \usepackage{selnolig}  % disable illegal ligatures
\fi

\title{Stat 590 Quiz 3}
\author{Tim Farkas}
\date{5/3/2021}

\begin{document}
\maketitle

The total credit of this quiz is 10 points. You must submit \textbf{your
own} lab as an knitted pdf file. To answer the questions, display the
results and write your comments if asked.

\begin{Shaded}
\begin{Highlighting}[]
\DocumentationTok{\#\# For reproducibility {-}{-}{-} don\textquotesingle{}t change this!}
\FunctionTok{set.seed}\NormalTok{(}\DecValTok{01182018}\NormalTok{)}
\FunctionTok{library}\NormalTok{(tidyverse)}
\end{Highlighting}
\end{Shaded}

\begin{verbatim}
## -- Attaching packages --------------------------------------- tidyverse 1.3.0 --
\end{verbatim}

\begin{verbatim}
## v ggplot2 3.3.2     v purrr   0.3.4
## v tibble  3.0.4     v dplyr   1.0.2
## v tidyr   1.1.2     v stringr 1.4.0
## v readr   1.4.0     v forcats 0.5.0
\end{verbatim}

\begin{verbatim}
## -- Conflicts ------------------------------------------ tidyverse_conflicts() --
## x dplyr::filter() masks stats::filter()
## x dplyr::lag()    masks stats::lag()
\end{verbatim}

\hypertarget{instructions}{%
\section{Instructions}\label{instructions}}

Maximization algorithm problems are required for everyone. Between the
EM algorithm and the Metropolis-Hasting algorithm problem, you may
choose one.

\hypertarget{maximization-algorithms}{%
\section{Maximization algorithms}\label{maximization-algorithms}}

Consider the density \(f(x) = [1-cos(x-\theta)]/2\pi\) on
\(0\leq x \leq 2\pi\), where \(\theta\) is a parameter between \(-\pi\)
and \(\pi\). The following i.i.d data arise from this density: 3.91,
4.85, 2.28, 4.06, 3.70, 4.04, 5.46, 3.53, 2.28, 1.96, 2.53, 3.88, 2.22,
3.47, 2.22, 3.47, 4.82, 2.46, 2.99, 2.54, 0.52, 2.50. We wish to
estimate \(\theta\).

\begin{itemize}
\tightlist
\item
  \textbf{1a.} (1 point) Write down the loglikelihood and graph the
  loglikelihood function.
\end{itemize}

\[Loglik(\theta) = - nlog(2\pi) + \sum_{i = 1}^nlog(1 - cos(x_i - \theta))\]
\[\sum-\frac{sin(x - \theta)}{[1 - cos(x - \theta)]}\]

\begin{Shaded}
\begin{Highlighting}[]
\NormalTok{ll}\FloatTok{.1}\NormalTok{a }\OtherTok{\textless{}{-}} \ControlFlowTok{function}\NormalTok{(x, theta) }\SpecialCharTok{{-}}\FunctionTok{length}\NormalTok{(x) }\SpecialCharTok{*} \FunctionTok{log}\NormalTok{(}\DecValTok{2} \SpecialCharTok{*}\NormalTok{ pi) }\SpecialCharTok{+} \FunctionTok{sum}\NormalTok{(}\FunctionTok{log}\NormalTok{(}\DecValTok{1} \SpecialCharTok{{-}} \FunctionTok{cos}\NormalTok{(x }\SpecialCharTok{{-}}\NormalTok{ theta)))}
\NormalTok{x }\OtherTok{\textless{}{-}} \FunctionTok{c}\NormalTok{(}\FloatTok{3.91}\NormalTok{, }\FloatTok{4.85}\NormalTok{, }\FloatTok{2.28}\NormalTok{, }\FloatTok{4.06}\NormalTok{, }\FloatTok{3.70}\NormalTok{, }\FloatTok{4.04}\NormalTok{, }\FloatTok{5.46}\NormalTok{, }\FloatTok{3.53}\NormalTok{, }\FloatTok{2.28}\NormalTok{, }\FloatTok{1.96}\NormalTok{, }\FloatTok{2.53}\NormalTok{, }\FloatTok{3.88}\NormalTok{, }\FloatTok{2.22}\NormalTok{, }\FloatTok{3.47}\NormalTok{, }\FloatTok{2.22}\NormalTok{, }\FloatTok{3.47}\NormalTok{, }\FloatTok{4.82}\NormalTok{, }\FloatTok{2.46}\NormalTok{, }\FloatTok{2.99}\NormalTok{, }\FloatTok{2.54}\NormalTok{, }\FloatTok{0.52}\NormalTok{, }\FloatTok{2.50}\NormalTok{)}
\NormalTok{y }\OtherTok{\textless{}{-}} 

\NormalTok{gg }\OtherTok{\textless{}{-}} \FunctionTok{tibble}\NormalTok{(}\AttributeTok{theta =} \FunctionTok{seq}\NormalTok{(}\SpecialCharTok{{-}}\NormalTok{pi, pi, }\AttributeTok{len =} \DecValTok{100}\NormalTok{), }
       \AttributeTok{y =} \FunctionTok{map\_dbl}\NormalTok{(}\FunctionTok{seq}\NormalTok{(}\SpecialCharTok{{-}}\NormalTok{pi, pi, }\AttributeTok{len =} \DecValTok{100}\NormalTok{), }
                   \SpecialCharTok{\textasciitilde{}} \FunctionTok{ll.1a}\NormalTok{(}\AttributeTok{x =}\NormalTok{ x, }\AttributeTok{theta =}\NormalTok{ .x))) }\SpecialCharTok{\%\textgreater{}\%}
  \FunctionTok{ggplot}\NormalTok{() }\SpecialCharTok{+} 
  \FunctionTok{geom\_line}\NormalTok{(}\FunctionTok{aes}\NormalTok{(}\AttributeTok{x =}\NormalTok{ theta, }\AttributeTok{y =}\NormalTok{ y))}

\FunctionTok{print}\NormalTok{(gg)}
\end{Highlighting}
\end{Shaded}

\includegraphics{quiz3_files/figure-latex/unnamed-chunk-2-1.pdf}

\begin{itemize}
\tightlist
\item
  \textbf{1b.} (1 point) Write down the bisection updating equation and
  write R codes to implement the bisection method, with starting values
  \(-2.7, -1, 0, 1, 2.7\).
\end{itemize}

\[[a_{t+1}, t_{t+1}] = \begin{array}{ll}
            [a_t, x^{(t)}]& \quad g'(a_t)g'(x^{(t)}) \le 0 \\
            [x^{(t)}, b_t]& \quad g'(a_t)g'(x^{(t)}) \gt 0
        \end{array}\] where

\[g'(x, \theta) = -\frac{sin(x - \theta)}{[1 - cos(x - \theta)]}\] and

\[x^{(t+1)} = (a_{t+1} + b_{t+1}) / 2\]

\begin{Shaded}
\begin{Highlighting}[]
\NormalTok{optim\_bisect }\OtherTok{\textless{}{-}} \ControlFlowTok{function}\NormalTok{(x, gprime, }\AttributeTok{start =} \SpecialCharTok{{-}}\DecValTok{1}\NormalTok{, }
                         \AttributeTok{left =} \SpecialCharTok{{-}}\NormalTok{pi, }\AttributeTok{right =}\NormalTok{ pi, }\AttributeTok{epsilon =}\NormalTok{ .}\DecValTok{01}\NormalTok{, }
                         \AttributeTok{adj =} \ConstantTok{FALSE}\NormalTok{) \{}
  
\NormalTok{  err }\OtherTok{=}\NormalTok{ epsilon }\SpecialCharTok{+} \DecValTok{1}
\NormalTok{  xt\_vec }\OtherTok{=}\NormalTok{ start}
  
  \ControlFlowTok{while}\NormalTok{(err }\SpecialCharTok{\textgreater{}}\NormalTok{ epsilon) \{}
    
\NormalTok{    slope }\OtherTok{\textless{}{-}} \FunctionTok{gprime}\NormalTok{(x, left) }\SpecialCharTok{*} \FunctionTok{gprime}\NormalTok{(x, xt\_vec[}\FunctionTok{length}\NormalTok{(xt\_vec)])}
    \ControlFlowTok{if}\NormalTok{(slope }\SpecialCharTok{\textless{}=} \DecValTok{0}\NormalTok{) \{ right }\OtherTok{\textless{}{-}}\NormalTok{ xt\_vec[}\FunctionTok{length}\NormalTok{(xt\_vec)] }
\NormalTok{    \} }\ControlFlowTok{else}\NormalTok{ left }\OtherTok{\textless{}{-}}\NormalTok{ xt\_vec[}\FunctionTok{length}\NormalTok{(xt\_vec)]}
\NormalTok{    xt\_vec[}\FunctionTok{length}\NormalTok{(xt\_vec) }\SpecialCharTok{+} \DecValTok{1}\NormalTok{] }\OtherTok{\textless{}{-}} \ControlFlowTok{if}\NormalTok{(adj) left }\SpecialCharTok{+}\NormalTok{ ( right }\SpecialCharTok{{-}}\NormalTok{ left) }\SpecialCharTok{/} \DecValTok{2} \ControlFlowTok{else}\NormalTok{ (left }\SpecialCharTok{+}\NormalTok{ right) }\SpecialCharTok{/} \DecValTok{2}
\NormalTok{    err }\OtherTok{\textless{}{-}} \FunctionTok{abs}\NormalTok{(xt\_vec[}\FunctionTok{length}\NormalTok{(xt\_vec)] }\SpecialCharTok{{-}}\NormalTok{ xt\_vec[}\FunctionTok{length}\NormalTok{(xt\_vec) }\SpecialCharTok{{-}} \DecValTok{1}\NormalTok{]) }\SpecialCharTok{/}
\NormalTok{      (}\FunctionTok{abs}\NormalTok{(xt\_vec[}\FunctionTok{length}\NormalTok{(xt\_vec)]) }\SpecialCharTok{+}\NormalTok{ epsilon)}
    
    
\NormalTok{  \}}
  
  \FunctionTok{return}\NormalTok{(xt\_vec)}
  
\NormalTok{\}}


\NormalTok{gprime }\OtherTok{\textless{}{-}} \ControlFlowTok{function}\NormalTok{(x, theta) }\FunctionTok{sum}\NormalTok{(}\SpecialCharTok{{-}}\FunctionTok{sin}\NormalTok{(x }\SpecialCharTok{{-}}\NormalTok{ theta) }\SpecialCharTok{/} 
\NormalTok{                                   (}\DecValTok{1} \SpecialCharTok{{-}} \FunctionTok{cos}\NormalTok{(x }\SpecialCharTok{{-}}\NormalTok{ theta)))}

\NormalTok{starts }\OtherTok{\textless{}{-}} \FunctionTok{c}\NormalTok{(}\SpecialCharTok{{-}}\FloatTok{2.7}\NormalTok{, }\SpecialCharTok{{-}}\DecValTok{1}\NormalTok{, }\DecValTok{0}\NormalTok{, }\DecValTok{1}\NormalTok{, }\FloatTok{2.7}\NormalTok{)}

\FunctionTok{map}\NormalTok{(starts, }\SpecialCharTok{\textasciitilde{}} \FunctionTok{optim\_bisect}\NormalTok{(}\AttributeTok{x =}\NormalTok{ x, }\AttributeTok{gprime =}\NormalTok{ gprime, }
             \AttributeTok{left =} \SpecialCharTok{{-}}\NormalTok{pi, }\AttributeTok{right =}\NormalTok{ pi, }\AttributeTok{start =}\NormalTok{ .x, }
             \AttributeTok{epsilon =} \FloatTok{0.001}\NormalTok{, }\AttributeTok{adj =} \ConstantTok{TRUE}\NormalTok{) }
\NormalTok{) }\SpecialCharTok{\%\textgreater{}\%}
  \FunctionTok{map\_dbl}\NormalTok{(}\SpecialCharTok{\textasciitilde{}}\NormalTok{ .x[}\FunctionTok{length}\NormalTok{(.x)])}
\end{Highlighting}
\end{Shaded}

\begin{verbatim}
## [1] -0.02573054 -0.02575946 -1.64289342 -0.02574502  2.87077216
\end{verbatim}

\begin{itemize}
\tightlist
\item
  \textbf{1c.} (1 point) Write down the Newton Rapson updating equation
  and write R codes to implement the method, with starting values
  \(-2.7, -1, 0, 1, 2.7\).
\end{itemize}

\[x^{(t+1)} = x^{(t)} - \frac{g'(x^{(t)})}{g''(x^{(t)})}\]

Where

\[g''(x, \theta) = 
\frac{1}{1 - cos(x - \theta)}\]

\begin{Shaded}
\begin{Highlighting}[]
\NormalTok{optim\_newtraph }\OtherTok{\textless{}{-}} \ControlFlowTok{function}\NormalTok{(dg, d2g, x, theta0, iter) \{ }
  
\NormalTok{  thetatrace }\OtherTok{\textless{}{-}}\NormalTok{ theta }\OtherTok{\textless{}{-}}\NormalTok{ theta0 }\CommentTok{\#initialize starting conditions}
  
  \DocumentationTok{\#\# iterate}
  \ControlFlowTok{for}\NormalTok{(i }\ControlFlowTok{in} \DecValTok{1}\SpecialCharTok{:}\NormalTok{iter) \{}
\NormalTok{    thetatrace[i}\SpecialCharTok{+}\DecValTok{1}\NormalTok{] }\OtherTok{\textless{}{-}}\NormalTok{ theta }\OtherTok{\textless{}{-}}\NormalTok{ theta }\SpecialCharTok{{-}} \FunctionTok{dg}\NormalTok{(x, theta)}\SpecialCharTok{/}\FunctionTok{d2g}\NormalTok{(x, theta)}
    \ControlFlowTok{if}\NormalTok{(}\FunctionTok{is.nan}\NormalTok{(theta)) \{}
\NormalTok{      thetatrace }\OtherTok{\textless{}{-}}\NormalTok{ thetatrace[}\SpecialCharTok{{-}}\FunctionTok{length}\NormalTok{(thetatrace)]}
\NormalTok{      theta }\OtherTok{\textless{}{-}}\NormalTok{ thetatrace[}\FunctionTok{length}\NormalTok{(thetatrace)]}
      \ControlFlowTok{break}
\NormalTok{    \}}
\NormalTok{  \}}
  \FunctionTok{return}\NormalTok{(}\FunctionTok{list}\NormalTok{(}\AttributeTok{final =}\NormalTok{ theta, }\AttributeTok{trace =}\NormalTok{ thetatrace)) }
  
\NormalTok{\}}

\NormalTok{g2prime }\OtherTok{\textless{}{-}} \ControlFlowTok{function}\NormalTok{(x, theta) }\FunctionTok{sum}\NormalTok{(}\DecValTok{1} \SpecialCharTok{/}\NormalTok{ (}\DecValTok{1} \SpecialCharTok{{-}} \FunctionTok{cos}\NormalTok{(x }\SpecialCharTok{{-}}\NormalTok{ theta))) }

\FunctionTok{map}\NormalTok{(starts, }\SpecialCharTok{\textasciitilde{}} \FunctionTok{optim\_newtraph}\NormalTok{(}\AttributeTok{dg =}\NormalTok{ gprime, }\AttributeTok{d2g =}\NormalTok{ g2prime, }
                             \AttributeTok{x =}\NormalTok{ x, }\AttributeTok{theta0 =}\NormalTok{ .x, }\AttributeTok{iter =} \DecValTok{100}\NormalTok{)}
\NormalTok{) }\SpecialCharTok{\%\textgreater{}\%} 
  \FunctionTok{map\_dbl}\NormalTok{(}\SpecialCharTok{\textasciitilde{}}\NormalTok{ .x}\SpecialCharTok{$}\NormalTok{final)}
\end{Highlighting}
\end{Shaded}

\begin{verbatim}
## [1] -2.753185 -1.463185  0.520000  2.220000  2.530000
\end{verbatim}

\begin{itemize}
\tightlist
\item
  \textbf{1d.} (1 point) Repeat part c using 200 equally spaced starting
  values between \(-\pi\) and \(\pi\).
\end{itemize}

\begin{Shaded}
\begin{Highlighting}[]
\FunctionTok{map}\NormalTok{(}\FunctionTok{seq}\NormalTok{(}\SpecialCharTok{{-}}\NormalTok{pi, pi, }\AttributeTok{len=}\DecValTok{200}\NormalTok{), }\SpecialCharTok{\textasciitilde{}} \FunctionTok{optim\_newtraph}\NormalTok{(}\AttributeTok{dg =}\NormalTok{ gprime, }\AttributeTok{d2g =}\NormalTok{ g2prime, }
                             \AttributeTok{x =}\NormalTok{ x, }\AttributeTok{theta0 =}\NormalTok{ .x, }\AttributeTok{iter =} \DecValTok{100}\NormalTok{)}
\NormalTok{) }\SpecialCharTok{\%\textgreater{}\%} 
  \FunctionTok{map\_dbl}\NormalTok{(}\SpecialCharTok{\textasciitilde{}}\NormalTok{ .x}\SpecialCharTok{$}\NormalTok{final) }\SpecialCharTok{\%\textgreater{}\%}
  \FunctionTok{tibble}\NormalTok{(}\AttributeTok{start =} \FunctionTok{seq}\NormalTok{(}\SpecialCharTok{{-}}\NormalTok{pi, pi, }\AttributeTok{len=}\DecValTok{200}\NormalTok{), }\AttributeTok{estimate =}\NormalTok{ .) }\SpecialCharTok{\%\textgreater{}\%}
  \FunctionTok{ggplot}\NormalTok{() }\SpecialCharTok{+} 
  \FunctionTok{geom\_line}\NormalTok{(}\FunctionTok{aes}\NormalTok{(}\AttributeTok{x =}\NormalTok{ start, }\AttributeTok{y =}\NormalTok{ estimate))}
\end{Highlighting}
\end{Shaded}

\includegraphics{quiz3_files/figure-latex/unnamed-chunk-5-1.pdf}

\begin{itemize}
\tightlist
\item
  \textbf{1e.} (1 point) Discuss your results from b-d.
\end{itemize}

This is an extremely lumpy likelihood with many local maxima, so it's
not surprising that the results are strongly dependent on starting
values. The difference between methods makes sense as well, since the
successive iterations in each will push the algorithm into different
zones, each with different optima, because there are SO many.

\hypertarget{em-algorithm}{%
\section{EM algorithm}\label{em-algorithm}}

Recall the peppered moth analysis introduced in Example 4.2. In the
field, it is quite difficult to distinguish the insularia and typica
pheotypes due to variations in wing color and mottle. In addition to the
622 moths mentioned in the example, suppose the sample collected by the
researchers actually included \(n_U= 578\) more moths that were known to
be \texttt{insularia} or \texttt{typical} but whose exact pheotypes
could not be determined.

\begin{itemize}
\tightlist
\item
  \textbf{2a.} (2.5 points) Derive the EM algorithm for maximum
  likelihood estimation of \(p_C, p_I,\) and \(p_I\) for this modified
  problem having observed data \(n_C, n_I, n_T\), and \(n_U\) as given
  above. The derivation include writing down the Q function, the E step,
  and the M step.
\end{itemize}

Expectations for \emph{carbonaria} are as in Example 4.2. Expectations
for \emph{insularia} and \emph{typica} should be \ldots{}

\[E(N_{II}|\mathbf{n}, \mathbf{p}) = n_{II}^{(t)} =
\frac{n_Ip_I^2}{p_I^2 + 2p_Ip_T} + 
\frac{n_Up_I^2}{p_I^2 + 2p_Ip_T + p_T^2}
\] \[E(N_{IT}|\mathbf{n}, \mathbf{p}) = n_{IT}^{(t)} = 
\frac{2n_Ip_Ip_T}{p_I^2 + 2p_Ip_T} + 
\frac{2n_Up_Ip_T}{p_I^2 + 2p_Ip_T + p_T^2}
\] \[E(N_{TT}|\mathbf{n}, \mathbf{p}) = n_{TT}^{(t)} = 
nT + 
\frac{n_Up_T^2}{p_I^2 + 2p_Ip_T + p_T^2}
\] \[Q(\mathbf{p}|\mathbf{p}^t) = 
n_{CC}^{(t)}log(p_C^2) +
n_{CI}^{(t)}log(2p_Cp_I) +
n_{CT}^{(t)}log(p_Cp_T) +
n_{II}^{(t)}log(p_I^2) +
n_{IT}^{(t)}log(2p_Ip_T) +
n_{TT}^{(t)}log(p_T^2) +
k(n_C, n_I, n_T, n_U, \mathbf{p}^{(t)})\]

The only differences here from example 4.2 is that \(n_{TT}\) is not
observed, because some TT alleles exist in the unknown samples, and the
conditional expectation for the multinomial coefficient depends also on
\(n_U\).

The M step is also the same as in the book. It's the expectations for
\(N_II\), \(N_IT\), and \(N_TT\) that differ.

\begin{itemize}
\tightlist
\item
  \textbf{2b.} (2.5 points) Write R codes to implement the algorithm.
\end{itemize}

\begin{Shaded}
\begin{Highlighting}[]
\CommentTok{\# ugh don\textquotesingle{}t look this is so ugly. }
\NormalTok{nCCt }\OtherTok{\textless{}{-}} \ControlFlowTok{function}\NormalTok{(nC, pC, pI, pT) nC}\SpecialCharTok{*}\NormalTok{pC}\SpecialCharTok{\^{}}\DecValTok{2} \SpecialCharTok{/}\NormalTok{ (pC}\SpecialCharTok{\^{}}\DecValTok{2} \SpecialCharTok{+} \DecValTok{2}\SpecialCharTok{*}\NormalTok{pC}\SpecialCharTok{*}\NormalTok{pI }\SpecialCharTok{+} \DecValTok{2}\SpecialCharTok{*}\NormalTok{pC}\SpecialCharTok{*}\NormalTok{pT)}
\NormalTok{nCIt }\OtherTok{\textless{}{-}} \ControlFlowTok{function}\NormalTok{(nC, pC, pI, pT) nC}\SpecialCharTok{*}\DecValTok{2}\SpecialCharTok{*}\NormalTok{pC}\SpecialCharTok{*}\NormalTok{pI }\SpecialCharTok{/}\NormalTok{ (pC}\SpecialCharTok{\^{}}\DecValTok{2} \SpecialCharTok{+} \DecValTok{2}\SpecialCharTok{*}\NormalTok{pC}\SpecialCharTok{*}\NormalTok{pI }\SpecialCharTok{+} \DecValTok{2}\SpecialCharTok{*}\NormalTok{pC}\SpecialCharTok{*}\NormalTok{pT)}
\NormalTok{nCTt }\OtherTok{\textless{}{-}} \ControlFlowTok{function}\NormalTok{(nC, pC, pI, pT) nC}\SpecialCharTok{*}\DecValTok{2}\SpecialCharTok{*}\NormalTok{pC}\SpecialCharTok{*}\NormalTok{pT }\SpecialCharTok{/}\NormalTok{ (pC}\SpecialCharTok{\^{}}\DecValTok{2} \SpecialCharTok{+} \DecValTok{2}\SpecialCharTok{*}\NormalTok{pC}\SpecialCharTok{*}\NormalTok{pI }\SpecialCharTok{+} \DecValTok{2}\SpecialCharTok{*}\NormalTok{pC}\SpecialCharTok{*}\NormalTok{pT)}
\NormalTok{nIIt }\OtherTok{\textless{}{-}} \ControlFlowTok{function}\NormalTok{(nI, nU, pI, pT) nI }\SpecialCharTok{*}\NormalTok{ pI}\SpecialCharTok{\^{}}\DecValTok{2} \SpecialCharTok{/}\NormalTok{ (pI}\SpecialCharTok{\^{}}\DecValTok{2} \SpecialCharTok{+} \DecValTok{2}\SpecialCharTok{*}\NormalTok{pT}\SpecialCharTok{*}\NormalTok{pI) }\SpecialCharTok{+}\NormalTok{ nU }\SpecialCharTok{*}\NormalTok{ pI}\SpecialCharTok{\^{}}\DecValTok{2} \SpecialCharTok{/}\NormalTok{ (pI}\SpecialCharTok{\^{}}\DecValTok{2} \SpecialCharTok{+} \DecValTok{2}\SpecialCharTok{*}\NormalTok{pI}\SpecialCharTok{*}\NormalTok{pT }\SpecialCharTok{+}\NormalTok{ pT}\SpecialCharTok{\^{}}\DecValTok{2}\NormalTok{)}
\NormalTok{nITt }\OtherTok{\textless{}{-}} \ControlFlowTok{function}\NormalTok{(nI, nU, pI, pT) nI }\SpecialCharTok{*} \DecValTok{2}\SpecialCharTok{*}\NormalTok{pI}\SpecialCharTok{*}\NormalTok{pT }\SpecialCharTok{/}\NormalTok{ (pI}\SpecialCharTok{\^{}}\DecValTok{2} \SpecialCharTok{+} \DecValTok{2}\SpecialCharTok{*}\NormalTok{pT}\SpecialCharTok{*}\NormalTok{pI) }\SpecialCharTok{+}\NormalTok{ nU }\SpecialCharTok{*} \DecValTok{2}\SpecialCharTok{*}\NormalTok{pI}\SpecialCharTok{*}\NormalTok{pT }\SpecialCharTok{/}\NormalTok{ (pI}\SpecialCharTok{\^{}}\DecValTok{2} \SpecialCharTok{+} \DecValTok{2}\SpecialCharTok{*}\NormalTok{pI}\SpecialCharTok{*}\NormalTok{pT }\SpecialCharTok{+}\NormalTok{ pT}\SpecialCharTok{\^{}}\DecValTok{2}\NormalTok{)}
\NormalTok{nTTt }\OtherTok{\textless{}{-}} \ControlFlowTok{function}\NormalTok{(nT, nU, pI, pT) nT }\SpecialCharTok{+}\NormalTok{ nU }\SpecialCharTok{*}\NormalTok{ pT}\SpecialCharTok{\^{}}\DecValTok{2} \SpecialCharTok{/}\NormalTok{ (pI}\SpecialCharTok{\^{}}\DecValTok{2} \SpecialCharTok{+} \DecValTok{2}\SpecialCharTok{*}\NormalTok{pI}\SpecialCharTok{*}\NormalTok{pT }\SpecialCharTok{+}\NormalTok{ pT}\SpecialCharTok{\^{}}\DecValTok{2}\NormalTok{)}

\NormalTok{pC\_next }\OtherTok{\textless{}{-}} \ControlFlowTok{function}\NormalTok{(nC, nI, nT, nU, pC, pI, pT, n) \{}
\NormalTok{  (}\DecValTok{2}\SpecialCharTok{*}\FunctionTok{nCCt}\NormalTok{(nC, pC, pI, pT) }\SpecialCharTok{+} \FunctionTok{nCIt}\NormalTok{(nC, pC, pI, pT) }\SpecialCharTok{+} \FunctionTok{nCTt}\NormalTok{(nC, pC, pI, pT))}\SpecialCharTok{/}\NormalTok{(}\DecValTok{2}\SpecialCharTok{*}\NormalTok{n)}
\NormalTok{\}}

\NormalTok{pI\_next }\OtherTok{\textless{}{-}} \ControlFlowTok{function}\NormalTok{(nC, nI, nT, nU, pC, pI, pT, n) \{}
\NormalTok{  (}\DecValTok{2}\SpecialCharTok{*}\FunctionTok{nIIt}\NormalTok{(nI, nU, pI, pT) }\SpecialCharTok{+} \FunctionTok{nITt}\NormalTok{(nI, nU,pI, pT) }\SpecialCharTok{+} \FunctionTok{nCIt}\NormalTok{(nC, pC, pI, pT))}\SpecialCharTok{/}\NormalTok{(}\DecValTok{2}\SpecialCharTok{*}\NormalTok{n)}
\NormalTok{\}}

\NormalTok{pT\_next }\OtherTok{\textless{}{-}} \ControlFlowTok{function}\NormalTok{(nC, nI, nT, nU, pC, pI, pT, n) \{}
\NormalTok{  (}\DecValTok{2}\SpecialCharTok{*}\FunctionTok{nTTt}\NormalTok{(nT, nU, pI, pT) }\SpecialCharTok{+} \FunctionTok{nCTt}\NormalTok{(nC, pC, pI, pT) }\SpecialCharTok{+} \FunctionTok{nITt}\NormalTok{(nI, nU,pI, pT))}\SpecialCharTok{/}\NormalTok{(}\DecValTok{2}\SpecialCharTok{*}\NormalTok{n)}
\NormalTok{\}}

\NormalTok{enns }\OtherTok{\textless{}{-}} \FunctionTok{tibble}\NormalTok{(}\AttributeTok{nC =} \DecValTok{85}\NormalTok{, }\AttributeTok{nI =} \DecValTok{196}\NormalTok{, }\AttributeTok{nT =} \DecValTok{341}\NormalTok{, }\AttributeTok{nU =} \DecValTok{622}\NormalTok{)}
\CommentTok{\#enns \textless{}{-} tibble(nC = 85, nI = 196, nT = 341, nU = 0) \# old }

\NormalTok{pees }\OtherTok{\textless{}{-}} \FunctionTok{tibble}\NormalTok{(}\AttributeTok{pC =} \FloatTok{0.33}\NormalTok{, }\AttributeTok{pI =} \FloatTok{0.33}\NormalTok{, }\AttributeTok{pT =} \FloatTok{0.33}\NormalTok{) }\SpecialCharTok{\%\textgreater{}\%} \FunctionTok{as.matrix}\NormalTok{()}
\NormalTok{i }\OtherTok{=} \DecValTok{1}
\NormalTok{Rcrit }\OtherTok{=} \FloatTok{0.001}
\NormalTok{R }\OtherTok{=}\NormalTok{ Rcrit }\SpecialCharTok{+} \DecValTok{1}
\ControlFlowTok{while}\NormalTok{(R }\SpecialCharTok{\textgreater{}}\NormalTok{ Rcrit) \{}
\NormalTok{  pees }\OtherTok{\textless{}{-}} \FunctionTok{rbind}\NormalTok{(pees, }\FunctionTok{c}\NormalTok{(}\FunctionTok{pC\_next}\NormalTok{(enns}\SpecialCharTok{$}\NormalTok{nC, enns}\SpecialCharTok{$}\NormalTok{nI, enns}\SpecialCharTok{$}\NormalTok{nT, enns}\SpecialCharTok{$}\NormalTok{nU, }
\NormalTok{                                pees[i, }\DecValTok{1}\NormalTok{], pees[i, }\DecValTok{2}\NormalTok{], pees[i, }\DecValTok{3}\NormalTok{], }\AttributeTok{n =} \FunctionTok{sum}\NormalTok{(}\FunctionTok{as.matrix}\NormalTok{(enns))), }
                        \FunctionTok{pI\_next}\NormalTok{(enns}\SpecialCharTok{$}\NormalTok{nC, enns}\SpecialCharTok{$}\NormalTok{nI, enns}\SpecialCharTok{$}\NormalTok{nT, enns}\SpecialCharTok{$}\NormalTok{nU, }
\NormalTok{                                pees[i, }\DecValTok{1}\NormalTok{], pees[i, }\DecValTok{2}\NormalTok{], pees[i, }\DecValTok{3}\NormalTok{], }\AttributeTok{n =} \FunctionTok{sum}\NormalTok{(}\FunctionTok{as.matrix}\NormalTok{(enns))), }
                        \FunctionTok{pT\_next}\NormalTok{(enns}\SpecialCharTok{$}\NormalTok{nC, enns}\SpecialCharTok{$}\NormalTok{nI, enns}\SpecialCharTok{$}\NormalTok{nT, enns}\SpecialCharTok{$}\NormalTok{nU, }
\NormalTok{                                pees[i, }\DecValTok{1}\NormalTok{], pees[i, }\DecValTok{2}\NormalTok{], pees[i, }\DecValTok{3}\NormalTok{], }\AttributeTok{n =} \FunctionTok{sum}\NormalTok{(}\FunctionTok{as.matrix}\NormalTok{(enns)))))}
\NormalTok{  i }\OtherTok{=}\NormalTok{ i}\SpecialCharTok{+}\DecValTok{1}
\NormalTok{  R }\OtherTok{=} \FunctionTok{sqrt}\NormalTok{(}\FunctionTok{sum}\NormalTok{((pees[i,] }\SpecialCharTok{{-}}\NormalTok{ pees[i }\SpecialCharTok{{-}} \DecValTok{1}\NormalTok{,])}\SpecialCharTok{\^{}}\DecValTok{2}\NormalTok{)) }\SpecialCharTok{/} \FunctionTok{sqrt}\NormalTok{(}\FunctionTok{sum}\NormalTok{(pees[i }\SpecialCharTok{{-}} \DecValTok{1}\NormalTok{,]))}
  
\NormalTok{\}}
\end{Highlighting}
\end{Shaded}

\hypertarget{metropolis-hasting-algorithm}{%
\section{Metropolis-Hasting
algorithm}\label{metropolis-hasting-algorithm}}

Simulating from a mixture distribution
\(\delta N(7, 0.5^2)+(1-\delta)N(10, 0.5^2)\) is straightforward.
However, using the Metropolis-Hasting algorithm to simulate realizations
from this distribution is useful for exploring the role of the proposal
distribution.

\begin{itemize}
\tightlist
\item
  \textbf{3a.} (1 point) Write down a Metropolis-Hasting algorithm to
  simulate from the mixture distribution with \(\delta=0.7\), using
  \(N(x^{(t)},0.01^2)\) as the proposal distribution.
\item
  \textbf{3b.} (2 points) Write R codes to implement the
  Metropolis-Hasting algorithm. For each of the three starting values,
  \(x^{(0)}=0, 7, 15\), run the chain for 10,000 iterations.
\item
  \textbf{3c.} (1 point) Plot the sample path of the output from each
  chain. What would you conclude about the chains? For each of the
  simulations, create a histogram of the realizations with the true
  density superimposed on the histogram. Based on the histograms, what
  can you say about the behavior of the chains?
\item
  \textbf{3d.} (1 point) Experiment on variance of the proposal
  distribution to improve the convergence properties of the chains.
\end{itemize}

\end{document}
